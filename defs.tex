\catcode`@=11

% PARAMETERS

\parskip=\smallskipamount
\parindent=0pt


% SHORTCUTS

\let\ea=\expandafter
\let\:=\colon
\let\dsty=\displaystyle
\let\tsty=\textstyle
\let\ssty=\scriptstyle
\let\sssty=\scriptscriptstyle


% FONTS

\font\ftitle=cmbx10 scaled \magstep2
\font\fprobtitle=cmbx10 scaled \magstep1
\font\fprobdesc=cmsl10 at 9pt

\def\loadfont#1#2#3#4#5{%
	\ea\font\csname ten#1\endcsname=#2
	\ea\font\csname seven#1\endcsname=#3
	\ea\font\csname five#1\endcsname=#4
	\ea\textfont\ea#5\ea=\csname ten#1\endcsname
	\ea\scriptfont\ea#5\ea=\csname seven#1\endcsname
	\ea\scriptscriptfont\ea#5\ea=\csname five#1\endcsname
	\ea\def\csname #1\endcsname{\fam#5}}

\newfam\bbfam
\loadfont{bb}{msbm10}{msbm7}{msbm5}\bbfam

\newfam\scrfam
\loadfont{scr}{stix-mathscr at 10pt}{stix-mathscr at 7pt}%
	{stix-mathscr at 5pt}\scrfam

\newfam\eucalfam
\loadfont{eucal}{eusm10}{eusm7}{eusm5}\eucalfam

\newfam\mbffam
\loadfont{mbf}{cmmib10}{cmmib7}{cmmib5}\mbffam
\skewchar\tenmbf='177
\skewchar\sevenmbf='177
\skewchar\fivembf='177


% MACROS

\def\1#1#2{#1} % I.e., firstoftwo
\def\2#1#2{#2} % I.e., secondoftwo

\def\hugeskip{\vskip 24pt plus 8pt\relax}

% Logic
\def\lxor{\oplus}
\def\limp{\rightarrow}
\def\Limp{\Rightarrow}
\def\liff{\leftrightarrow}
\def\Liff{\Leftrightarrow}

% Sets
\def\N{{\bb N}}
\def\Z{{\bb Z}}
\def\Q{{\bb Q}}
\def\R{{\bb R}}
\def\powset{{\scr P}}

% Big O and friends
% bigomega and bigtheta are upright, so bigo should be too, hence eucal
% (but not just a roman O, since it looks too much like zero).
% \Omega and \Theta are class 7, so make sure they get typeset in \rm.
% \omega is class 0, so no worries there.
\def\bigo{{\eucal O}}
\def\littleo{{\mit o}}
\def\bigomega{{\rm \Omega}}
\def\littleomega{\omega}
\def\bigtheta{{\rm \Theta}}

\def\ihat{{\bf\mathaccent"705E \mathchar"7010}}
\def\jhat{{\bf\mathaccent"705E \mathchar"7011}}
\def\khat{{\bf\mathaccent"705E k}}

\def\?{\mathrel{\mathop=\limits^?}}
\def\d{{\rm d}} % For dx in integrals, etc. Preceed with \, after \int.
\def\falling#1#2{#1^{\underline{#2}}}
\def\rising#1#2{#1^{\overline{#2}}}

% https://www.tug.org/TUGboat/tb22-4/tb72perlS.pdf
\def\clap#1{\hbox to 0pt{\hss#1\hss}}
\def\mathllap{\mathpalette\@mathllapinternal}
\def\mathrlap{\mathpalette\@mathrlapinternal}
\def\mathclap{\mathpalette\@mathclapinternal}
\def\@mathllapinternal#1#2{\llap{$\mathsurround=0pt#1{#2}$}}
\def\@mathrlapinternal#1#2{\rlap{$\mathsurround=0pt#1{#2}$}}
\def\@mathclapinternal#1#2{\clap{$\mathsurround=0pt#1{#2}$}}

% mappalette is like mathpalette, except instead of the first argument to the
% given CS being the current math style, it is whatever the given mapfour maps
% the current math style to.
\def\mapfour#1#2#3#4#5{\ifcase #5 #1\or #2\or #3\or #4\fi}
\def\mappalette#1#2#3{\mathchoice
	{\edef\@arg{#10}\ea#2\ea{\@arg}{#3}}{\edef\@arg{#11}\ea#2\ea{\@arg}{#3}}
	{\edef\@arg{#12}\ea#2\ea{\@arg}{#3}}{\edef\@arg{#13}\ea#2\ea{\@arg}{#3}}}

% Superimposition
\def\supimp#1#2{{%
	\leavevmode
	\setbox1=\hbox{#1}\setbox2=\hbox{#2}
	\ifdim\wd1<\wd2
		\setbox0=\box1 \setbox1=\box2 \setbox2=\box0
	\fi
	\hbox to \wd1{\hfil\unhbox2\hfil}\llap{\unhbox1}}}
\def\@msupimpinternal#1#2{%
	\supimp{$\mathsurround=0pt \1#1\1#2$}{$\mathsurround=0pt \2#1\2#2$}}
\def\msupimpmap#1#2#3{\mappalette{#1}\@msupimpinternal{{#2}{#3}}}
\def\msupimp{\msupimpmap
	{\mapfour{\dsty\dsty}{\tsty\tsty}{\ssty\ssty}{\sssty\sssty}}}

\long\def\problem#1\par#2\endproblem{{%
	\par\vfil\penalty0\vfilneg\bigskip
	\hbox{\fprobtitle #1}%
	\nobreak\medskip
	\setbox0=\vbox{
		\advance \hsize by -30pt
		\fprobdesc\noindent #2}%
	\dimen0=\ht0
	\dimen1=\dp0
	\advance \dimen0 by 2pt
	\advance \dimen1 by 2pt
	\hbox to \hsize{
		\hfil
		\vrule height \dimen0 depth \dimen1
		\hskip 5pt
		\box0
		\hskip 5.4pt plus 1fil}
	\penalty25\smallskip}}

% Spread tokens in #1 by glue in #2.
\def\spread#1#2{{%
	\def\endhelp##1\help{}
	\def\help##1{##1\hskip#2\relax\help}
	\help#1\endhelp\unskip}}

\catcode`@=11
\endinput

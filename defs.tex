\catcode`@=11

\def\supershowlists{{\showboxdepth=1000 \showboxbreadth=1000000 \showlists}}


% PARAMETERS/REGISTER ALLOCATION

% \parindent, prob@sidesize, item@indent, and pc@sideskip should
% be kept in sync so that indentations look consistent.

\parindent=0pt
\parskip=\smallskipamount

% TODO allocate registers for temporary use
\newskip\hugeskipamount    \hugeskipamount=24pt plus8pt minus8pt
\newdimen\prob@sidesize    \prob@sidesize=20pt
\newdimen\prob@rulegap     \prob@rulegap=5pt
\newdimen\prob@topskip     \prob@topskip=\bigskipamount
\newdimen\item@indent      \item@indent=20pt
\newcount\pc@topbotpenalty \pc@topbotpenalty=-200
\newcount\pc@stmtpenalty   \pc@stmtpenalty=100
\newcount\pc@clubpenalty   \pc@clubpenalty=250
\newcount\pc@widowpenalty  \pc@widowpenalty=250
\newcount\pc@interpenalty  \pc@interpenalty=10000
\newdimen\pc@hang          \pc@hang=30pt
\newdimen\pc@indent        \pc@indent=20pt
\newdimen\pc@linenogap     \pc@linenogap=5pt
\newskip\pc@sideskip       \pc@sideskip=20pt
\newskip\pc@topbotskip     \pc@topbotskip=\bigskipamount


% FONTS

% We use 10/7/5 for normal text, and 9/6/5 for problem statement text.
% Little/no effort has been made to make sure math symbols work properly with
% non-standard fonts (but the Latin/Greek letters should work fine).
% Plain TeX default fams, for reference:
%  0 = rm             (10/7/5)
%  1 = math it        (10/7/5)
%  2 = math symbols   (10/7/5)
%  3 = math extension (10/10/10)
%  4 = text it        (10//)
%  5 = sl             (10//)
%  6 = bf             (10/7/5)
%  7 = tt             (10//)

% Misc fonts
\font\bigbf=cmbx12 scaled \magstep1 % Title.
\font\twelvebf=cmbx12 % Problem headings.

\font\ninerm=cmr9
\font\sixrm=cmr6

\font\ninei=cmmi9 \skewchar\ninei='177
\font\sixi=cmmi6  \skewchar\sixi='177

\font\ninesy=cmsy9 \skewchar\ninesy='60
\font\sixsy=cmsy6  \skewchar\sixsy='60

\font\nineit=cmti9

\font\ninesl=cmsl9

\font\ninebf=cmbx9
\font\sixbf=cmbx6

\font\ninett=cmtt9

\font\tensc=cmcsc10
\font\ninesc=cmcsc9

\font\tenss=cmss10
\font\niness=cmss9
% \font\sevenss=cmss7
% \font\sixss=cmss6
% \font\fivess=cmss5

\font\tenbb=msbm10
\font\ninebb=msbm9
\font\sevenbb=msbm7
\font\sixbb=msbm6
\font\fivebb=msbm5

\font\tenscr=stix-mathscr at 10pt
\font\ninescr=stix-mathscr at 9pt
\font\sevenscr=stix-mathscr at 7pt
\font\sixscr=stix-mathscr at 6pt
\font\fivescr=stix-mathscr at 5pt

\font\teneucal=eusm10
\font\nineeucal=eusm9
\font\seveneucal=eusm7
\font\sixeucal=eusm6
\font\fiveeucal=eusm5

\font\tenbmit=cmmib10  \skewchar\tenbmit='177
\font\ninebmit=cmmib9  \skewchar\ninebmit='177
\font\sevenbmit=cmmib7 \skewchar\sevenbmit='177
\font\sixbmit=cmmib6   \skewchar\sixbmit='177
\font\fivebmit=cmmib5  \skewchar\fivebmit='177

\newfam\scfam
\newfam\ssfam
\newfam\bbfam
\newfam\scrfam
\newfam\eucalfam
\newfam\bmitfam

\def\initfam#1#2#3#4{%
	\textfont#1=#2%
	\scriptfont#1=#3%
	\scriptscriptfont#1=#4}

% These 4 switches, along with \mit (fam1) and \cal (fam2) don't change
% the current font, only the fam, so they don't need to be redefed in \tenpt
% or \ninept.
\def\bb{\fam=\bbfam}
\def\scr{\fam=\scrfam}
\def\eucal{\fam=\eucalfam}
\def\bmit{\fam=\bmitfam}

\def\tenpt{%
	\initfam0\tenrm\sevenrm\fiverm
	\initfam1\teni\seveni\fivei
	\initfam2\tensy\sevensy\fivesy
	\initfam\itfam\tenit\nullfont\nullfont
	\initfam\slfam\tensl\nullfont\nullfont
	\initfam\bffam\tenbf\sevenbf\fivebf
	\initfam\ttfam\tentt\nullfont\nullfont
	\initfam\scfam\tensc\nullfont\nullfont
	\initfam\ssfam\tenss\nullfont\nullfont
	\initfam\bbfam\tenbb\sevenbb\fivebb
	\initfam\scrfam\tenscr\sevenscr\fivescr
	\initfam\eucalfam\teneucal\seveneucal\fiveeucal
	\initfam\bmitfam\tenbmit\sevenbmit\fivebmit
	\def\rm{\fam=0      \tenrm}%
	\def\it{\fam=\itfam \tenit}%
	\def\sl{\fam=\slfam \tensl}%
	\def\bf{\fam=\bffam \tenbf}%
	\def\tt{\fam=\ttfam \tentt}%
	\def\sc{\fam=\scfam \tensc}%
	\def\ss{\fam=\ssfam \tenss}%
	\setbox\strutbox=\hbox{\vrule height8.5pt depth3.5pt width0pt}%
	\normalbaselineskip=12pt
	\normalbaselines\rm}

\def\ninept{%
	\initfam0\ninerm\sixrm\fiverm
	\initfam1\ninei\sixi\fivei
	\initfam2\ninesy\sixsy\fivesy
	\initfam\itfam\nineit\nullfont\nullfont
	\initfam\slfam\ninesl\nullfont\nullfont
	\initfam\bffam\ninebf\sixbf\fivebf
	\initfam\ttfam\ninett\nullfont\nullfont
	\initfam\scfam\ninesc\nullfont\nullfont
	\initfam\ssfam\niness\nullfont\nullfont
	\initfam\bbfam\ninebb\sixbb\fivebb
	\initfam\scrfam\ninescr\sixscr\fivescr
	\initfam\eucalfam\nineeucal\sixeucal\fiveeucal
	\initfam\bmitfam\ninebmit\sixbmit\fivebmit
	\def\rm{\fam=0      \ninerm}%
	\def\it{\fam=\itfam \nineit}%
	\def\sl{\fam=\slfam \ninesl}%
	\def\bf{\fam=\bffam \ninebf}%
	\def\tt{\fam=\ttfam \ninett}%
	\def\sc{\fam=\scfam \ninesc}%
	\def\ss{\fam=\ssfam \niness}%
	\setbox\strutbox=\hbox{\vrule height8pt depth3pt width0pt}%
	\normalbaselineskip=11pt
	\normalbaselines\rm}

\tenpt


% MACROS/ASSIGNMENTS

\def\1#1#2{#1} % I.e., firstoftwo
\def\2#1#2{#2} % I.e., secondoftwo

% Shortcuts
\let\ea=\expandafter
\let\dsty=\displaystyle
\let\tsty=\textstyle
\let\ssty=\scriptstyle
\let\sssty=\scriptscriptstyle
\let\lc=\lceil
\let\rc=\rceil
\let\lf=\lfloor
\let\rf=\rfloor

% Alternate names
\let\:=\colon % Punctuation colon. Regular colon is a relation.
\let\lxor=\oplus
\let\limp=\rightarrow
\let\liff=\leftrightarrow
\let\grad=\nabla
\let\del=\partial

% Vertical skips
\def\hugeskip{\vskip\hugeskipamount}
\def\xpvskip#1#2{\par\ifdim\lastskip<#2\removelastskip\penalty#1\vskip#2\fi}
\def\xvskip#1{\xpvskip0{#1}}
\def\xsmallskip{\xvskip\smallskipamount}
\def\xmedskip{\xvskip\medskipamount}
\def\xbigskip{\xvskip\bigskipamount}
\def\xhugeskip{\xvskip\hugeskipamount}

% \imp to match \iff.
\def\imp{\;\Longrightarrow\;}

% Sets
\def\N{{\bb N}}
\def\Z{{\bb Z}}
\def\Q{{\bb Q}}
\def\R{{\bb R}}
\def\powset{{\scr P}}

% Big O and friends
% bigomega and bigtheta are upright, so bigo should be too, hence eucal
% (but not just a roman O, since it looks too much like zero).
% \Omega and \Theta are class 7, so make sure they get typeset in \rm.
% \omega is class 0, so no worries there.
\def\bigo{{\eucal O}}
\def\littleo{{\mit o}}
\def\bigomega{{\rm \Omega}}
\def\littleomega{\omega}
\def\bigtheta{{\rm \Theta}}

\def\ihat{{\bf\mathaccent"705E \mathchar"7010}}
\def\jhat{{\bf\mathaccent"705E \mathchar"7011}}
\def\khat{{\bf\mathaccent"705E k}}

\def\?{\mathrel{\mathop=\limits^?}}
\def\d{{\rm d}} % For dx in integrals, etc. Preceed with \, usually.
\def\falling#1#2{#1^{\underline{#2}}}
\def\rising#1#2{#1^{\overline{#2}}}
\def\rect#1#2#3{% inner width, inner height, line thickness 
	\vcenter{\vbox{%
		\hrule height#3
		\hbox{\vrule width#3 height#2 \kern#1 \vrule width#3}%
		\hrule height#3}}}
\def\qedbox{\rect{5pt}{5pt}{0.5pt}}
\def\qed{\unskip\hskip0.5em plus1fil \hbox{}\nobreak\hfill$\qedbox$\par}

% https://www.tug.org/TUGboat/tb22-4/tb72perlS.pdf
\def\clap#1{\hbox to 0pt{\hss#1\hss}}
\def\mathllap{\mathpalette\@mathllapinternal}
\def\mathrlap{\mathpalette\@mathrlapinternal}
\def\mathclap{\mathpalette\@mathclapinternal}
\def\@mathllapinternal#1#2{\llap{$\mathsurround=0pt#1{#2}$}}
\def\@mathrlapinternal#1#2{\rlap{$\mathsurround=0pt#1{#2}$}}
\def\@mathclapinternal#1#2{\clap{$\mathsurround=0pt#1{#2}$}}

% mappalette is like mathpalette, except instead of the first argument to the
% given CS being the current math style, it is whatever the given mapfour maps
% the current math style to.
\def\mapfour#1#2#3#4#5{\ifcase #5 #1\or #2\or #3\or #4\fi}
\def\mappalette#1#2#3{\mathchoice
	{\edef\@arg{#10}\ea#2\ea{\@arg}{#3}}{\edef\@arg{#11}\ea#2\ea{\@arg}{#3}}
	{\edef\@arg{#12}\ea#2\ea{\@arg}{#3}}{\edef\@arg{#13}\ea#2\ea{\@arg}{#3}}}

% Superimposition
% This was mostly an excercise. This would probably
% be better done with alignment macros like ooalign.
\def\supimp#1#2{{%
	\leavevmode
	\setbox1=\hbox{#1}\setbox2=\hbox{#2}%
	\ifdim\wd1<\wd2
		\setbox0=\box1 \setbox1=\box2 \setbox2=\box0
	\fi
	\hbox to \wd1{\hfil\unhbox2\hfil}\llap{\unhbox1}}}
\def\@msupimpinternal#1#2{%
	\supimp{$\mathsurround=0pt \1#1\1#2$}{$\mathsurround=0pt \2#1\2#2$}}
\def\msupimpmap#1#2#3{\mappalette{#1}\@msupimpinternal{{#2}{#3}}}
\def\msupimp{\msupimpmap
	{\mapfour{\dsty\dsty}{\tsty\tsty}{\ssty\ssty}{\sssty\sssty}}}

% TODO allow breaking problem statements across pages.
\long\def\problem#1\par#2\endproblem{{%
	\par
	\vskip 0pt plus100pt \penalty0 \vskip 0pt plus-100pt
	\vskip\prob@topskip
	\nointerlineskip\hbox{\twelvebf#1}%
	\nobreak\medskip
	\setbox0=\vbox{%
		\advance \hsize by -2\prob@sidesize
		\def\nostrut{\let\strut=\relax}%
		\ninept\sl\noindent\strut#2\strut}%
	\dimen0=\ht0 \advance \dimen0 by 2pt
	\dimen1=\dp0 \advance \dimen1 by 2pt
	\nointerlineskip\hbox to \hsize{%
		\hfil
		\vrule height\dimen0 depth\dimen1
		\hskip\prob@rulegap
		\box0
		\hskip\prob@rulegap
		\hskip 0.4pt plus1fil}%
	\nointerlineskip
	\penalty25\medskip\vskip-\parskip}}

\let\itemitem=\undefined
\def\item{\@item1}
\def\iitem{\@item2}
\def\iiitem{\@item3}
\def\@item#1#2{%
	\par \hangafter=0 \hangindent=#1\item@indent
	\noindent\llap{\rm#2\enspace}\ignorespaces}

% Spread tokens in #1 by glue in #2.
\def\spread#1#2{%
	\def\@terminator##1\@help{}%
	\def\@help##1{##1\hskip#2\relax\@help}%
	\@help#1\@terminator\unskip}

\def\initverb{%
	\catcode`\\=12 \catcode`\{=12
	\catcode`\}=12 \catcode`\$=12
	\catcode`\&=12 \catcode`\#=12
	\catcode`\^=12 \catcode`\_=12
	\catcode`\%=12 \catcode`\~=12
	\obeyspaces\obeylines}
\def\verb#1#2{\begingroup
	#1% Arbitrary local initialization before catcodes change.
	\initverb
	\long\def\@verb##1#2{##1\endgroup}%
	\@verb}

% Zero pad a positive integer. E.g., \zeropad{000}{13} -> 013.
% Idea from https://tex.stackexchange.com/a/412225/202281.
\def\zeropad#1#2{{%
	\ifnum1#1>1#2%
		\zeropad{#1}{0#2}%
	\else
		#2%
	\fi}}

% From the TeXBook TOC.
\def\diamondleaders{%
	\ifodd\pc@lineno \kern-10pt \fi
	\leaders\hbox to 20pt{%
		\ifodd\pc@lineno \kern15pt \else \kern5pt \fi $\cdot$\hss}}

% Same as plain raggedright, but 2em -> 3em, and it advances rightskip,
% instead of setting it.
\def\raggedright{%
	\advance \rightskip by 0pt plus 3em
	\spaceskip=0.3333em \xspaceskip=0.5em\relax}

% TODO alignment support
\newcount\pc@penalty
\newcount\pc@lineno
\newbox\pc@comment
\def\pc@kwskip{\hskip0.5em\relax}
\def\pc@linenofmt{00}
\def\beginpcode{%
	\par\begingroup
	\parskip=0pt
	\advance \leftskip by \pc@sideskip
	\advance \rightskip by \pc@sideskip
	\raggedright
	\clubpenalty=0 \widowpenalty=0
	\interlinepenalty=\pc@interpenalty
	\hyphenpenalty=10000 \exhyphenpenalty=10000
	\mathsurround=0pt
	\everypar={\pc@state}%
	\setbox0=\hbox{$\ssty\pc@linenofmt$\kern\pc@linenogap}%
	\parindent=\wd0
	\pc@penalty=0
	\pc@lineno=0
	\setbox\pc@comment=\box\voidb@x
	\xpvskip\pc@topbotpenalty\pc@topbotskip
	\pc@dorule\kern2pt}
\def\endpcode{%
	\kern2pt \pc@dorule
	\xpvskip\pc@topbotpenalty\pc@topbotskip
	\endgroup\vskip-\parskip}
\def\pc@addpenalty#1{\advance \pc@penalty by #1}
\def\pc@emitpenalty{%
	\ifnum\pc@penalty>0
		\penalty\pc@penalty
		\pc@penalty=0
	\fi}
\def\pc@dorule{%
	\nointerlineskip
	\hbox to \hsize{%
		\hskip\leftskip
		\kern-2pt
		\dimen255=\hsize
		\advance \dimen255 by -\leftskip
		\advance \dimen255 by -\rightskip
		\advance \dimen255 by 4pt
		\vrule height0.4pt width\dimen255
		\kern-2pt
		\hskip\rightskip}%
	\nointerlineskip}
\def\pc@enter{\advance \parindent by \pc@indent}
\def\pc@exit{\advance \parindent by -\pc@indent}
\def\pc@state#1;{%
	\pc@emitpenalty
	\hangafter=1
	\hangindent=\parindent \advance \hangindent by \pc@hang
	\advance \pc@lineno by 1
	\llap{\hbox to \parindent{%
		$\ssty\zeropad{\pc@linenofmt}{\number\pc@lineno}$\hss}}%
	% The struts assure the surrounding rules are spaced well.
	\strut#1\strut
	\ifvoid\pc@comment\else
		\nobreak\hfil\hbox{}\hskip0.5em plus1fil \hbox{}\nobreak\hfill
		\box\pc@comment% pc@comment now void.
	\fi
	\par
	\pc@addpenalty\pc@stmtpenalty}
\def\State{\leavevmode}
\def\Func#1(#2){%
	\Kw{function}\pc@kwskip\Name{#1}(#2);
	\pc@addpenalty\pc@clubpenalty
	\pc@enter}
\def\If#1\Then{%
	\Kw{if}\pc@kwskip #1\unskip\pc@kwskip\Kw{then};
	\pc@addpenalty\pc@clubpenalty
	\pc@enter}
\def\Elseif#1\Then{%
	\pc@exit
	\pc@addpenalty\pc@widowpenalty
	\Kw{elseif}\pc@kwskip #1\unskip\pc@kwskip\Kw{then};
	\pc@addpenalty\pc@clubpenalty
	\pc@enter}
\def\Else{%
	\pc@exit
	\pc@addpenalty\pc@widowpenalty
	\Kw{else};
	\pc@addpenalty\pc@clubpenalty
	\pc@enter}
\def\For$#1$#2\From$#3$#4\To$#5$#6\Do{%
	\Kw{for}\pc@kwskip$#1 := #3$\pc@kwskip
	\Kw{to}\pc@kwskip$#5$\unskip\pc@kwskip\Kw{do};
	\pc@addpenalty\pc@clubpenalty
	\pc@enter}
\def\Foreach#1\In#2\Do{%
	\Kw{foreach}\pc@kwskip #1\unskip\pc@kwskip
	\Kw{in}\pc@kwskip #2\unskip\pc@kwskip\Kw{do};
	\pc@addpenalty\pc@clubpenalty
	\pc@enter}
\def\While#1\Do{%
	\Kw{while}\pc@kwskip #1\unskip\pc@kwskip\Kw{do};
	\pc@addpenalty\pc@clubpenalty
	\pc@enter}
\def\End{%
	\pc@exit
	\pc@addpenalty\pc@widowpenalty
	\Kw{end};}
\def\Return{\Kw{return}\pc@kwskip}
\def\Kw#1{\leavevmode{\bf #1}}
\def\Id#1{\leavevmode{\it #1}}
\def\Name#1{\leavevmode{\sc #1}}
\def\Call#1(#2){\leavevmode\hbox{\Name{#1}(#2)}}
\def\And{\unskip\pc@kwskip\Kw{and}\pc@kwskip}
\def\Or{\unskip\pc@kwskip\Kw{or}\pc@kwskip}
\def\True{{\rm true}}
\def\False{{\rm false}}
\def\Comment#1{\setbox\pc@comment=\hbox{$\triangleright$ #1}}
\def\Mark#1{\edef#1{\the\pc@lineno}}
\def\Markg{\global\Mark}

\catcode`@=12
\endinput
